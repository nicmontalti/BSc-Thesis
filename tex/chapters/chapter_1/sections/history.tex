%\section*{Polarons: Historical overview}
The term polaron was first used by Solomon Pekar in 1946 to define an electron that localizes itself in a potential well, self-generated by the polarization of the material \cite{pekar1946}. The result can be described as an electron surrounded by a cloud of phonons. The electron moves in the crystal and the polarization follows it as it moves.

Lev Landau was the first to propose the concept of an auto-localized electron in a crystal in a 1933 paper \cite{terhaar1965}. The idea was then developed by Pekar, who considered a single electron interacting with a dielectric continuum medium \cite{pekar1946, pekar1947}. This interaction was shown to cause an enhancement of the effective mass and a localization of the wavefunction \cite{landau1948}. In their work, Landau and Pekar used a quantum mechanical description of the electron and a classical description of the medium. A full quantum mechanical description was then developed by Fr\"ohlich \cite{frohlich1950} and Holstein \cite{holstein1959}, who formalized the distinction between large and small polarons.

Fr\"ohlich considered an electron in a continuum, polarizable medium. The electron is assumed to interact only with longitudinal optical phonons. The interaction gives rise to the polarization of the material, which generates a potential well in which the electron localizes. Since the medium is treated as a continuum, the result is valid only for large polarons, that are polarons with an effective radius larger than the lattice constant. On the other hand, Holstein considered short-range electron-phonon interactions, resulting from the coupling between a carrier and the strain where it resides. Holstein theory takes into account the discreteness of the lattice, and it is used to describe small polarons, that are polarons with an effective radius smaller than the lattice constant.

All the attempts find analytical solutions to Fr\"ohlich Hamiltonian have been fruitless, and Holstein Hamiltonian is exactly solvable only in the two-site case \cite{rongsheng2002}. Approximation techniques and numerical simulations are unavoidable. Good results have been achieved with the diagrammatic quantum Monte Carlo method to solve both the Fr\"ohlich and Holstein Hamiltonians \cite{prokofev1998,mishchenko2000}. For small polarons, DFT+U methods have also proven to be applicable \cite{kokott2018}. A rigo­rous, ab initio computational theory of polarons was recently developed by Feliciano Giustino and colleagues combining the Landau–Pekar model with DFT \cite{sio2019}.

\section*{Outline}
The aim of this thesis is to give an introductory analytical description of polarons and to perform a numerical simulation of a small polaron in rutile. The focus of the analytical discussion is to derive Holstein and Fröhlich Hamiltonians. The results of the numerical simulation are discussed at a qualitative level.

Polarons originate from the coupling of electrons with phonons in crystals. To give a complete description of these quasi-particles is necessary to introduce a mathematical formalism that allows us to describe such a big number of particles and their interactions. Second quantization proved to be a good choice for various many-body problems in solid state physics, polarons included.

In \cref{ch:review}, we will start by reviewing the second quantization formalism, and we will use it to describe electrons and phonons in crystals. Electrons and phonons are firstly studied as non-interacting systems. Then, their interaction, essential to describe polarons, is investigated as well.

In \cref{ch:polarons} the theory of polarons is explained starting from the original Landau-Pekar model. Then, Fr\"ohlich and Holstein Hamiltonians are derived and solved in the small and large coupling regimes using perturbation theory and variational principles. Eventually, small and large polarons are compared.

In \cref{ch:dft} Density Functional Theory (DFT) is presented. The theory behind it is explained together with some of its problems. An extension of it, DFT+U, is also introduced to solve some inaccuracies. Lastly, its implementation is discussed, focusing on VASP, the Vienna Ab-initio Simulation Package.

In \cref{ch:simulation} the simulation of a small polaron in $\ce{TiO_2}$ is presented. The whole simulation process is described, and the results are discussed. The polaronic solution is compared to a delocalized one, with particular emphasis one the density of states, band structure and charge isosurfaces.