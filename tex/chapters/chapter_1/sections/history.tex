\section{Historical overview}
Lev Landau was the first to propose the concept of an auto-localized in a crystal in a 1933 paper \cite{terhaar1965}. The idea was developed by Pekar, who considered a single electron interacting with a dielectric medium \cite{pekar1946, pekar1947}. This interaction was shown to cause an enhancement of the effective mass and a localization of the wavefunction \cite{landau1948}. The Landau-Pekar model gives a quantum mechanical description of the electron and a classical description of the medium. A full quantum mechanical description was then developed by Fr\"ohlich \cite{frohlich1950} and Holstein \cite{holstein1959}, who formalized the distinction between large and small polarons. A description of these models is given in \cref{sec:small_large}.