Lev Landau was the first to propose the concept of an auto-localized electron in a crystal in a 1933 paper \cite{terhaar1965}. The idea was then developed by Pekar in 1946, who considered a single electron interacting with a dielectric continuum medium \cite{pekar1946, pekar1947}. He was the first to use the term \emph{polaron} to define an electron that localizes itself in a potential well self-generated by the polarization of the material. This interaction was shown to cause an enhancement of the effective mass and a localization of the wavefunction \cite{landau1948}. In their work, Landau and Pekar used a quantum mechanical description of the electron and a classical description of the medium. A full quantum mechanical description was then developed by Fröhlich \cite{frohlich1950} and Holstein \cite{holstein1959}, who formalized the distinction between large and small polarons.

Fröhlich considered an electron in a continuum, polarizable medium. In his model, the electron is assumed to interact only with longitudinal optical phonons. The interaction gives rise to the polarization of the material, which generates a potential well in which the electron localizes. Since the medium is treated as a continuum, the results are only valid for large polarons, that are polarons with an effective radius larger than the lattice constant. On the other hand, Holstein considered short-range electron-phonon interactions, resulting from the coupling between a carrier and the strain where it resides. Holstein theory takes into account the discreteness of the lattice, and it is used to describe small polarons, for which the effective radius is smaller than the lattice constant.

All the attempts to find analytical solutions to the Fröhlich Hamiltonian have been fruitless, whereas the Holstein Hamiltonian is exactly solvable only in the two-site case \cite{rongsheng2002}. Approximation techniques and numerical simulations are then unavoidable. Good results in solving both the Fröhlich and Holstein Hamiltonians have been achieved with the Diagrammatic Quantum Monte Carlo method \cite{prokofev1998,mishchenko2000}. For small polarons, DFT+U methods have also proven to be applicable \cite{kokott2018}. A rigo­rous, ab initio computational theory of polarons was recently developed by Feliciano Giustino and colleagues combining the Landau–Pekar model with DFT \cite{sio2019}.

\section*{Outline}
The aim of this thesis is to give an introductory analytical description of polarons and to perform a numerical simulation of a small polaron in rutile. The focus of the analytical discussion is to derive Holstein and Fröhlich Hamiltonians. The numerical calculation aims to simulate a small polaron in rutile $\ce{TiO_2}$.

In \cref{ch:review}, we introduce the mathematical formalism and physical laws necessary for the study of polarons. We start by reviewing the second quantization formalism, and use it to describe electrons and phonons in crystals. Electrons and phonons are firstly studied as non-interacting particles. Then, the electron-phonon interaction, essential for the description of polarons, is investigated as well.

In \cref{ch:polarons}, the theory of polarons is explained starting by the original Landau-Pekar model. Then, Fröhlich and Holstein Hamiltonians are derived and solved in the small and large coupling regimes. Lastly, small and large polarons main properties are discussed and compared.

In \cref{ch:dft}, Density Functional Theory (DFT) is presented. The theory behind it is explained together with some of its problems. An extension of it, DFT+U, is also introduced to solve some inaccuracies. Lastly, its implementation is discussed focusing on the Vienna Ab-initio Simulation Package (VASP).

In \cref{ch:simulation}, we present a simulation of a small polaron in $\ce{TiO_2}$. The whole simulation process is described, and the results are discussed. The polaronic solution is compared with an electron delocalized in the material, with particular emphasis on the density of states, band structure and charge isosurfaces.