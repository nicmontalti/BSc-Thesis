\section{Vienna Ab-initio Simulation Package}
VASP is a computer program  developed by the University of Vienna for DFT calculations \cite{kresse1993,kresse1996,kresse1996a,kresse1999}. This software has been used for all the simulations performed in this thesis. In this section, we will describe how a general simulation in VASP is performed.

Before performing a calculation, the basic elements that VASP needs are: the position of the atoms in the crystal, the k-space mesh grid to be used, and the pseudopotentials. Each of this element is given in input to the program through a different file. The first file is named POSCAR, and it contains a description of the crystal. An example is given below
%\begin{minipage}{\linewidth}
\lstinputlisting{POSCAR_example}
%\end{minipage}
The first half of the file describes the lattice. The lattice is defined through the three basis vectors $\vec{a}_1$, $\vec{a}_2$ and $\vec{a}_3$. These vectors are given by multiplying the lattice constants defined in line 2 with the vectors defined in lines 4-6. The second half of the file describes the composition of the primitive cell. Lines 7 and 8 define the type and number of atoms in the cell, whereas line 12-13 their positions. The positions can be expressed in cartesian coordinates (Cartesian) or in the lattice basis $\vec{a}_1$, $\vec{a}_2$, $\vec{a}_3$ (Direct).

The second file is named KPOINTS, and it describes the k-space grid. The grid can be automatically generated by VASP or given explicitly by the user. An example of the first case is given below
%\begin{minipage}{\linewidth}
\lstinputlisting{KPOINTS_example}
%\end{minipage}
The zero in line 2 indicates that the grid has to be generated automatically with the generation method specified in line 3. Line 4 gives the number of points that have to generated in every direction and line 5 an optional shift with respect to the centre of the Brillouin zone. In our example, a $\Gamma$-centered $11\times11\times11$ grid is generated.

In band structure calculations, it is convenient to specify a custom grid. Instead of calculating the energy of the electrons for evenly-spaced k-points, some specific k-points of interest are chosen. Typically, they are special points of symmetry, like the centre of the Brillouin zone or the center of a side of the Fermi surface. An example is given below.
\lstinputlisting{KPOINTS_band_example}
In this example, three lines of 10 k-points are generated. The first goes from the point $\Gamma$ to $X$, the second from $X$ to $W$ and the third from $W$ to $\Gamma$. The number of points per line is defined in line 2. Like in the POSCAR, the points coordinates may be expressed in cartesian coordinates (Cartesian) or with respect to the reciprocal base $\vec{b}_1$, $\vec{b}_2$, $\vec{b}_3$ (Reciprocal).

The last file that defines the system is named POTCAR. It contains all the information about the pseudopotentials that have to be used for the atoms in the cell. If more than one type of atom is present, multiple POTCARs are concatenated in a unique file.

All the computational instructions, like the energy cut-off, are specified in the INCAR file. The INCAR file can contain a very vast number of parameters, of which giving a comprehensive description here would be impossible. We will  present some of them in the next chapter, when we need to use them. A complete list of the tags that can be used in the INCAR file can be found in the VASP wiki \cite{zotero-172}.

%Finally, many output files are produced, among which there are: the OUTCAR, that contains all the details of the computation; the WAVECAR, that contains the final wavefunctions; the CHGCAR, that contains the final charge density. These last two files are particularly important if we want to start a simulation in the state in which another one ended.