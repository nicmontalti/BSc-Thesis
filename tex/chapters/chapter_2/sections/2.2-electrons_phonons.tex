\section{Electrons and phonons in a crystal}
In this section we will briefly describe how free electrons and phonons behave in a crystal, starting with a description of a crystal lattice. The interaction between two electrons and two phonons will not be considered. The electron-phonon interaction will be discussed in the following section.
\subsection{Crystal lattice}
Solid state physics deals with materials made of huge numbers of atoms, of the order of the Avogadro number. We have just developed a mathematical formalism (second quantization) with which it is possible to treat such systems. However, it is clearly impossible to solve the equations for a general N-body system. Luckily, X-ray diffraction experiments showed that many solids exhibit particular symmetry properties. The most important is translation symmetry.

%ADD BETTER DEFINITION OF LATTICE

It was discovered that many materials, called crystals, can be described as an organized collection of atoms sitting in a lattice, usually called Bravais lattice. A lattice is a collection of points translationally invariant. This means that given any two points described by two vectors $\vec{R_1}$ and $\vec{R_2}$, the difference between them is
\begin{equation}
    \vec{R}_1 - \vec{R}_2 = \vec{R}_n
\end{equation}
with $\vec{R}_n = n_1\vec{a}_1 + n_2\vec{a}_2 + n_3\vec{a}_3$. The three vectors $\vec{a}_1, \vec{a}_2, \vec{a}_3$ are called basis vectors and $n_1, n_2, n_3$ are integers. Every property $f(\vec{r})$ of the lattice is then invariant under a translation of a lattice vector $\vec{R}_n$
\begin{equation}
    f(\vec{r}+\vec{R}_n) = f(\vec{r})
\end{equation}
We will see that applying this principle to the potential generated by the ions of the crystal will have important implications on the description of the electrons.

Associated to the Bravais lattice, there is a second one called reciprocal lattice. It is defined by three other base vectors $\vec{b}_1, \vec{b}_2, \vec{b}_3$, with
\begin{equation} \label{eq:reciprocal_base}
    \vec{b}_1 = \frac{2\pi}{V} \ \vec{a}_2 \times \vec{a}_3 \quad
    \vec{b}_2 = \frac{2\pi}{V} \ \vec{a}_3 \times \vec{a}_1
    \quad
    \vec{b}_3 = \frac{2\pi}{V} \ \vec{a}_1 \times \vec{a}_2
\end{equation}
where $V = \vec{a}_1 \cdot \vec{a}_2 \cross \vec{a}_3$. A vector in the reciprocal lattice is usually written as $\vec{G}_m = m_1\vec{a}_1 + m_2\vec{b}_2 + m_3\vec{b}_3$. From \cref{eq:reciprocal_base} it is easy to see that the product of a base Bravais lattice vector $\vec{a}_i$ and a base reciprocal lattice vector $\vec{b}_j$ is
\begin{equation}
    \vec{a}_i \cdot \vec{b}_j = 2\pi \delta_{ij}
\end{equation}

\subsection{Electrons in crystals}