\section{Electrons and phonons in a crystal}
In this section we will briefly describe how free electrons and phonons behave in a crystal, starting with a description of a crystal lattice. The interaction between two electrons and two phonons will not be considered. The electron-phonon interaction will be discussed in the following section.
\subsection{Crystal lattice}
Solid state physics deals with materials made of huge numbers of atoms, of the order of the Avogadro number. We have just developed a mathematical formalism (second quantization) with which it is possible to treat such systems. However, it is clearly impossible to solve the equations for a general N-body system. Luckily, X-ray diffraction experiments showed that many solids exhibit particular symmetry properties, useful to simplify the problem.

Many solids, called crystals, are composed by the repetition in space of an identical unit cell. The unit cell is defined as the smallest repeating unit having the full symmetry of the crystal structure \cite{westBasicSolidState1999}.
Each unit cell is place on a point of a Bravais lattice. The Bravais lattice, also referred to as space lattice, describes the geometric arrangement of the lattice points \cite{lernerEncyclopediaPhysicsVolumes2005}. Given any two points of the lattice, described by the vectors $\vec{R}_1$ and $\vec{R}_2$, the difference between them is
\begin{equation}
    \vec{R}_1 - \vec{R}_2 = \vec{R}_n
\end{equation}
with $\vec{R}_n = n_1\vec{a}_1 + n_2\vec{a}_2 + n_3\vec{a}_3$. The three vectors $\vec{a}_1, \vec{a}_2, \vec{a}_3$ are called basis vectors and $n_1, n_2, n_3$ are integers. The position of a single atom in the crystal can then be expressed as the position in the unit cell $\vec{\tau}_aj$ plus the position of the lattice point $\vec{R}_n$
\begin{equation}
    \vec{R}_n^a = \vec{R}_n + \vec{\tau}_a
\end{equation}
Given the symmetry of the system, every property $f(\vec{r})$ of the lattice is invariant under a translation of a lattice vector $\vec{R}_n$
\begin{equation} \label{eq:translational_invariance}
    f(\vec{r}+\vec{R}_n) = f(\vec{r})
\end{equation}
We will see that applying this principle to the potential generated by the ions of the crystal will have important implications on the description of the electrons.

Associated to the Bravais lattice, there is a second one called reciprocal lattice. It is defined by three other basis vectors $\vec{b}_1, \vec{b}_2, \vec{b}_3$, with
\begin{equation} \label{eq:reciprocal_basis}
    \vec{b}_1 = \frac{2\pi}{\Omega} \ \vec{a}_2 \times \vec{a}_3
    \qquad
    \vec{b}_2 = \frac{2\pi}{\Omega} \ \vec{a}_3 \times \vec{a}_1
    \qquad
    \vec{b}_3 = \frac{2\pi}{\Omega} \ \vec{a}_1 \times \vec{a}_2
\end{equation}
where $\Omega = \vec{a}_1 \cdot \vec{a}_2 \cross \vec{a}_3$ is the volume of a unit cell. A vector in the reciprocal lattice is usually written as $\vec{G}_m = m_1\vec{a}_1 + m_2\vec{b}_2 + m_3\vec{b}_3$, where $m_1, m_2, m_3$ are integers. From \cref{eq:reciprocal_basis} it is easy to see that the product of a basis Bravais lattice vector $\vec{a}_i$ and a basis reciprocal lattice vector $\vec{b}_j$ is
\begin{equation}
    \vec{a}_i \cdot \vec{b}_j = 2\pi \delta_{ij}
\end{equation}

\subsection{Electrons in crystals}
The simplest model of electrons in a solid is the Sommerfield model. It was developed by Arnold Sommerfield in 1928 \cite{sommerfeldZurElektronentheorieMetalle1928}, combining the Drude model \cite{drude1900a} with Fermi-Dirac statistics. The electrons are treated as quantum non-interacting free particles. The electrons are treated as quantum non-interacting free particles, which implies that the wavefunctions are plane waves
\begin{equation} \label{eq:plane_wave}
    \psi_\vec{k} = e^{i\vec{k}\cdot\vec{r}}
\end{equation}
The energy is entirely kinetic, thus the dispersion relation is
\begin{equation} \label{eq:free_dispersion}
    \epsilon_\vec{k} = \frac{\hbar k^2}{2m}
\end{equation}

Despite its simplicity, this model is surprisingly good at describing a vast number of physical phenomena. Examples are the Wiedemann–Franz law \cite{jones1985}, electrons heat capacity and electrical conductivity. However, it does not give any explanation for the different properties of conductors, insulators and semiconductors.

To correctly describe the properties of electrons, we have to take into account the potential generated by the ions in the crystal. The interaction is entirely electrical, so the potential is the well known Coulomb potential
\begin{equation}
    V_{e-N} = \sum_{i=1}^{N_e} \sum_{j=1}^{N_N} \frac{1}{4\pi\epsilon_0} \frac{-Z_je^2}{|\vec{r}_i - \vec{R}_j|}
\end{equation}
where the first sum is extended on all the electrons and the second on the nuclei. It is convenient to divide the electrons in inner core electrons and valence electrons. The formers are tightly bound to the nucleus and occupy closed inner shells. They do not interact with other atoms of the crystal, so the nucleus together with its core electrons can be treated as a positive ion. The valence electrons belong to  non-closed shells and form chemical bonds with other atoms. Despite this description of electrons being apparently simpler, the potential of interaction between valence electrons and ions cannot be treated as simple Coulomb potential anymore.

To overcome the complexity of solving a many-body \sche   with a long range electromagnetic interaction, we leverage the symmetry of the crystal. Recalling our previous discussion, we know the potential of the ions to be translationally invariant. Recalling \cref{eq:translational_invariance}
\begin{equation}
    V(\vec{r}+\vec{R}_n) = V(\vec{r})
\end{equation}
The \sche for a periodic potential $V$ is then
\begin{equation}
    \oper{H}_\text{Bloch} \Psi = \left(-\frac{\hbar^2}{2m} \laplacian + V \right)\Psi = \epsilon\Psi
\end{equation}
Block proved in 1929 \cite{bloch1929} that the solutions of this problem are the Bloch functions $\Psi_{n\vec{k}}$:
\begin{equation}
    \oper{H}_\text{Block} \Psi_{n\vec{k}} = \epsilon_n(\vec{k}) \Psi_{n\vec{k}}
    \quad
    \rightarrow
    \quad
    \Psi_{n\vec{k}}(\vec{r}) = u_{n\vec{k}}(\vec{r}) e^{i\vec{k}\cdot\vec{r}}
\end{equation}
where $u$ is a function with the same periodicity of $V$, $\vec{k}$ is a wavevector and $n$ is the band index. The plane wave solution showed in \cref{eq:plane_wave} is a simple case where $V$ and $u$ are constant. An important consequence of the Bloch theorem is that the solutions to the \sche, even if they are not plane waves, can still be indexed with the quantum number $\vec{k}$, along with the band index $n$.

Expanding $\epsilon_n(\vec{k})$ around $\vec{k} = 0$, for an isotropic energy band,
\begin{equation} \label{eq:effective_mass}
    \epsilon_n(\vec{k}) = \epsilon_n(0) + \frac{\hbar k^2}{2m^*} + \mathcal{O}(k^2)
\end{equation}
where $m^*$ is the effective mass, defined as
\begin{equation}
    \frac{1}{m^*} = \frac{1}{\hbar^2} \pdv[2]{\epsilon}{k}
\end{equation}
Expression \labelcref{eq:effective_mass} is formally identical to \cref{eq:free_dispersion} with $m^*$ in place of $m$. For small values of $k$ electrons can then be treated as free particles of mass $m^*$.
The final non-interacting electron hamiltonian can be rewritten in second quantization as
\begin{equation}
    \oper{H}_\text{el} = \sum_{n\vec{k}} \epsilon_{n\vec{k}} \adjop{c}_{n\vec{k}}\oper{c}_{n\vec{k}}
\end{equation}