\section{Holstein polarons}
Holstein proposed a different Hamiltonian to model polarons. The main difference is in how the lattice is treated. Instead of working with a continuum, polarizable medium like Fröhlich, Holstein took into account the discreteness of the lattice.

The model is based on the tight-binding model, which we will briefly introduce in the next session. We will then proceed to derive Holstein Hamiltonian for a 1D system, and solve it in the weak coupling limit.

\subsection{Tight binding model} \label{sec:tight-binding}
In solid state physics, the most simple models used to describe electrons are the \emph{nearly-free electrons model} and the \emph{tight-binding model}. These two models approach the problem from an almost opposite point of view. If in the \emph{nearly-free electrons model} electrons are treated as free particles, perturbed by the ionic potential, in the \emph{tight-binding model} electrons are treated as tightly bounded to the atoms.

Since the electrons are tightly bounded to the atoms, we suppose their wavefunction to be a linear combination of the atomic orbitals. Atomic orbitals $\phi_{ta}$ are functions that satisfy the equation
\begin{equation}
    \left(\frac{\vec{p}^2}{2m} + V_a(\vec{r}) \right) \phi_{ta} = E_{ta}\phi_{ta}
\end{equation}
where $V_a$ is the potential generated by an ion of type $a$ and $t$ is a label for different atomic states. We from an LCAO which will be used as basis to expand the electron wavefunction
\begin{equation} \label{eq:lcao_tight}
    \phi_\vec{k}^{ta}(\vec{r}) = \frac{1}{\sqrt[]{N}}\sum_{n} e^{i\vec{k}\cdot\vec{r}}\phi_{ta}(\vec{r}-\vec{R}_n - \vec{\tau}_a)
\end{equation}
where $N$ is the number of unit cells. The factor $e^{i\vec{k}\cdot\vec{r}}$ ensures that $\psi$ is of the Bloch form.

In the tight-binding approximation, an electron wavefunction is expressed as a linear combination of the atomic orbitals in \cref{eq:lcao_tight}:
\begin{equation} \label{eq:lcao}
    \psi_\vec{k}(\vec{r}) = \sum_{ta} c_{ta}(\vec{k}) \phi_\vec{k}^{ta}(\vec{r})
\end{equation}
We could now solve the \sche, but it is more useful trying a different approach and working in second quantization.

We start from a system of free electrons, using \cref{eq:electron_hamiltonian} we can write the Hamiltonian as
\begin{equation}
    \oper{H}_\text{free} = \sum_{\vec{k}} \epsilon_{\vec{k}} \adjop{c}_{\vec{k}}\oper{c}_{\vec{k}}
\end{equation}
where
\begin{equation} \label{eq:tight_free_disperion}
    \epsilon_{\vec{k}} = \frac{\hbar\vec{k}^2}{2m}
\end{equation}
The creation and annihilation operator in the momentum space are related to the respective creation and annihilation operators in position space by a Fourier transform
\begin{align} \label{eq:fourier_creation}
    \adjop{c}_\vec{k} & = \frac{1}{\sqrt{N}} \sum_j e^{i\vec{k}\cdot\vec{r}_j} \adjop{c}_j \\ \label{eq:fourier_annihilation}
    \oper{c}_\vec{k}  & = \frac{1}{\sqrt{N}} \sum_j e^{-i\vec{k}\cdot\vec{r}_j} \oper{c}_j
\end{align}
Using \cref{eq:fourier_creation,eq:fourier_annihilation} we can write the Hamiltonian in position space
\begin{equation} \label{eq:tight_hamiltonian_position}
    \oper{H}_\text{free} = \frac{1}{N} \sum_{ij} \sum_\vec{k} \epsilon_{\vec{k}} e^{i\vec{k}\cdot(\vec{r}_i - \vec{r}_j)} \adjop{c}_i\oper{c_j}
\end{equation}
where $N$ is the number of available $\vec{k}$ states. It is easy to interpret the effect of the creation and annihilation operators in \cref{eq:tight_hamiltonian_position}: an electron moves from $\vec{r}_j$ to $\vec{r}_i$ and
\begin{equation}
    \tilde{t}_{ij} = \sum_\vec{k} \epsilon_{\vec{k}} e^{i\vec{k}\cdot(\vec{r}_i - \vec{r}_j)}
\end{equation}
is the associated kinetic energy.

If we now consider a system of non-interacting electrons moving in a Bravais lattice, so subject to an period ionic potential, the dispersion relation in \cref{eq:tight_free_disperion} will change. The factor $\tilde{t}_{ij}$ will change too. We will refer to the new factor $t_{ij}$ as the \emph{hopping parameter}. It can be interpreted as the change in energy after an electron moves from the site $j$ to the site $i$. The result is that the electrons will tend to become more localized, since the value of $t_{ij}$ will be very small for large distances $|\vec{r}_i - \vec{r}_j|$. In the tight-binding approximation, we assume
\begin{equation} \label{eq:tight_hamiltonian_position}
    t_{ij} =
    \begin{cases}
        -t \qquad                                     & \text{for nearest neighbors} \\
        0                                      \qquad & \text{otherwise}
    \end{cases}
\end{equation}
The tight-binding Hamiltonian becomes
\begin{equation}
    \oper{H}_\text{tb} = -t \sum_{\langle i,j \rangle} (\adjop{c}_i\oper{c_j} +  \adjop{c}_j\oper{c_i})
\end{equation}
where ${\langle i,j \rangle}$ means that the sum is extended only over the $(i,j)$ that are nearest neighbors.

It is useful to express \cref{eq:tight_hamiltonian_position} in the momentum-space representation. In order to do so, we rewrite \cref{eq:tight_hamiltonian_position} as
\begin{equation}
    \oper{H}_\text{tb} = -t \sum_{\langle i,j \rangle} (\adjop{c}_i\oper{c_j} +  \adjop{c}_j\oper{c_i})
    = -\frac{t}{2} \sum_{i} \sum_\delta (\adjop{c}_i\oper{c_{i+\delta}} +  \adjop{c}_{i+\delta}\oper{c_i})
\end{equation}
where the sum over $\delta$ is carried over all the nearest neighbors of the site $i$ and the factor $1/2$ is inserted to avoid double counting. Using \cref{eq:fourier_creation,eq:fourier_annihilation}, we can express the Hamiltonian in the momentum-space representation:
\begin{multline}
    \oper{H}_\text{tb}
    = -\frac{t}{2} \frac{1}{N} \sum_{i} \sum_\delta \sum_{\vec{k}, \vec{k}'}
    (e^{-i\vec{k}\cdot\vec{r}_i} e^{i\vec{k}'\cdot(\vec{r}_i+\vec{r}_\delta)} \adjop{c}_\vec{k} \oper{c_{\vec{k}'}} +
    e^{i\vec{k}\cdot\vec{r}_i} e^{-i\vec{k}'\cdot(\vec{r}_i+\vec{r}_\delta)} \adjop{c}_\vec{k} \oper{c_{\vec{k}'}}) \\
    = -\frac{t}{2} \sum_{\vec{k},\delta} (e^{i\vec{k}\cdot\vec{r}_\delta}+e^{-i\vec{k}\cdot\vec{r}_\delta})  \adjop{c}_\vec{k} \oper{c_{\vec{k}}}
    = -t \sum_{\vec{k},\delta} \cos(\vec{k}\cdot\vec{r}_\delta) \adjop{c}_\vec{k} \oper{c_{\vec{k}}}
\end{multline}
which can be written as
\begin{equation} \label{eq:tight_hamiltonian_momentum}
    \oper{H}_\text{tb} = \sum_{\vec{k}} \epsilon_{\vec{k}}^\text{tb} \adjop{c}_{\vec{k}}\oper{c}_{\vec{k}}
\end{equation}
with
\begin{equation}
    \epsilon_\vec{k}^\text{tb} = -t \sum_{    \delta} \cos(\vec{k}\cdot\vec{r}_\delta)
\end{equation}

\subsection{Derivation of the Holstein Hamiltonian}
We now use the tight-binding model to sketch the derivation of the Holstein Hamiltonian. This Hamiltonian is used to model small polarons, since it treats the polarizable medium as a lattice.

To simplify the derivation, we consider a linear chain of $N$ atoms. At equilibrium, the atom $n$ is placed at the position $R_n = na$, where $a$ is the lattice constant. The atoms are allowed to move, and their interaction energy is given by the harmonic oscillator approximation. The Hamiltonian of the lattice is then
\begin{equation}
    \oper{H_{I-I}} = \sum_n \left( \frac{{P}_n^2}{2M} + \oh M\omega_0^2x_n^2 \right)
\end{equation}
where $P_n$ is the momentum of the $n^\text{th}$ atom, $x_n$ the separation between the atoms $n$ and $n+1$ and $M$ the mass.

The electrons are allowed to interact with the lattice through a potential
\begin{equation}
    U = \sum_n U(r-R_n, x_n)
\end{equation} The key feature of this model is that $U$ depends on the interatomic separation $x_n$. This results in the coupling of the electrons with the lattice vibrations, so with phonons. Adding the electron kinetic energy term, the total Hamiltonian is given by
\begin{equation}
    \oper{H} = \sum_n \left( -\frac{\hbar^2}{2M} \frac{\partial^2}{\partial x_n^2} + \oh M\omega_0^2x_n^2 \right)
    -\frac{\hbar^2}{2m} \frac{\partial^2}{\partial r^2}
    +     \sum_n U(r-R_n, x_n)
\end{equation}
Following the tight-binding approach described in the previous section, we express the electrons wavefunctions as a linear combination of the single atomic orbitals
\begin{equation}
    \psi(r) = \sum_{n} \alpha_{n}(x_1 \dots x_N) \phi_n(r-na, x_n)
\end{equation}
where $\alpha_n$ complex coefficients and $\phi_n$ are the solution of the corresponding \sche
\begin{equation}
    \left[ -\frac{\hbar^2}{2m} \frac{\partial^2}{\partial r^2} + U(r-na, x_n) \right] \phi_n = E_n(x_n) \phi_n
\end{equation}

This result in a differential equation for the coefficients $\alpha_n$, which is obtained through a standard projection procedure. The full calculation can be found in the Appendix of the original Holstein paper \cite{holstein1959}. We only report the result
\begin{multline} \label{eq:holstein_diff}
    \left[ i\hbar \pdv{}{t} - \sum_p \left( -\frac{\hbar^2}{2M} \pdv[2]{}{x_p} + \oh M \omega_0^2 x_p^2 \right) - E(x_n) - W_n(x_1\dots x_N) \right] \alpha_n(x_1\dots x_N)
    \\= \sum_{(\pm)} J(x_n, x_{n+1}) \alpha_{n\pm1}(x_1\dots x_N)
\end{multline}
where
\begin{align}
    W_n(x_1 \dots x_N) & = \int |\phi_n(r-na, x_n)|^2  \sum_{p \neq n} U(r-pa, x_p)  \differential r \\
    J(x_n, x_m)        & = \int \phi_n^*(r-na, x_n) U(r-na, x_n) \phi(r-ma, x_m) \differential r
\end{align}

We can simplify \cref{eq:holstein_diff} introducing three approximations:
\begin{enumerate}
    \item The neglect of the energies $ W_n(x_1 \dots x_N)$.
    \item The neglect of the $x$-dependance of $J$. The function $J(x_n, x_m)$ reduces to a constant $-J$, which can also be denominated $-t$. It corresponds to the parameter $t$ introduced in the tight-binding approximation in \cref{eq:tight_hamiltonian_position}.
    \item The $x$-dependance of the energy $E_n(x_n)$ is taken to be linear: $    E_n(x_n) = - A x_n$
\end{enumerate}
The physical meaning of the first approximation (1) is to neglect the perturbation of the wavefunction localized on the site $n$ caused by the interaction with other sites $p$. This assumption is reasonable in the tight-binding model that we are employing. The result is that the expectation value of the energy of an electron is only dependent on one coordinate $x_n$, the one of the site where the electron is localized.

With the previous simplifications, \cref{eq:holstein_diff} becomes
\begin{equation} \label{eq:holstein_diff_simp}
    i\hbar\pdv{}{t} \alpha_n = \sum_p\left( -\frac{\hbar^2}{2M} \pdv[2]{}{x_p} + \oh M \omega_0^2 x_p^2 \right) \alpha_n - t (\alpha_{n+1}+\alpha_{n-1})\alpha_n
    - A x_n \alpha_n
\end{equation}
A study of \cref{eq:holstein_diff_simp} in its current form is presented by Holstein in his paper \cite{holstein1959}. However, we will follow a different approach, expressing it in the second-quantization formalism. From now on, we also consider again a 3D system, generalizing the results that we obtain for the 1D one.

The first term of \cref{eq:holstein_diff_simp} is exactly the lattice interaction term we have already encountered in \cref{eq:lattice_hamiltonian}. In the momentum space, its second-quantized form is given by \cref{eq:phonon_hamiltonian}
\begin{equation}
    \oper{H}_\text{ph} = \hbar \omega_0\sum_{\vec{q}}  \left( \adjop{b}_{\vec{q}} \oper{b}_{\vec{q}} + \frac{1}{2} \right)
\end{equation}
where we have assumed the phonons to be dispersionless and optical, as we did for the Fröhlich Hamiltonian. The second term is the tight-binding energy encountered in \cref{sec:tight-binding} and its second-quantized form is give by \cref{eq:tight_hamiltonian_momentum}
\begin{equation}
    \oper{H}_\text{el} = \sum_{\vec{k}} \epsilon_{\vec{k}}^\text{tb} \adjop{c}_{\vec{k}}\oper{c}_{\vec{k}}
\end{equation}
Finally, the last term is responsible for the electron-phonon interaction energy. We have seen that, for a normal scattering process, the general form of the interaction matrix is given by \cref{eq:phonon_matrix_normal}. Since in this case the interaction amplitude is independent of the momentum, we can write the interaction matrix as constant $g$ in the position space and $g / \sqrt{N}$ in the momentum space. The interaction Hamiltonian is then given by \cref{eq:general_electron_phonon_hamiltonian}
\begin{equation}
    \oper{H}_\text{el-ph} = \frac{g}{\sqrt{N}}\sum_{\vec{k}, \vec{q}} \adjop{c}_{\vec{k}+\vec{q}} \oper{c}_\vec{k} (\adjop{b}_{-\vec{q}} + \oper{b}_\vec{q})
\end{equation}
The final Holstein Hamiltonian is
\begin{equation}
    \oper{H}_\text{Holstein} = \sum_{\vec{k}} \epsilon_{\vec{k}}^\text{tb} \adjop{c}_{\vec{k}}\oper{c}_{\vec{k}} + \hbar \omega_0\sum_{\vec{q}}  \left( \adjop{b}_{\vec{q}} \oper{b}_{\vec{q}} + \frac{1}{2} \right) + \frac{g}{\sqrt{N}}\sum_{\vec{k}, \vec{q}} \adjop{c}_{\vec{k}+\vec{q}} \oper{c}_\vec{k} (\adjop{b}_{-\vec{q}} + \oper{b}_\vec{q})
\end{equation}

\subsection{Weak coupling limit}
Like the Fröhlich Hamiltonian, the Holstein Hamiltonian is not exactly solvable in its general form. Although it is exactly solvable for a two state system \cite{tayebi2016}, we will use perturbation theory to find an approximated solution. We will follow the same procedure we adopted in \cref{sec:weak_frohlich}. This method is valid as long as the coupling constant
\begin{equation}
    \alpha = \frac{g^2}{3\hbar\omega_0t}
\end{equation}
is small.

We start by splitting the Holstein Hamiltonian in a unperturbed term $\oper{H}^{(0)} = \oper{H}_\text{el} + \oper{H}_\text{ph}$ and the perturbation $\oper{H}_\text{el-ph}$.
The solution to the unperturbed problem is found solving the Hamiltonian
\begin{equation}
    \oper{H^{(0)}} = \sum_{\vec{k}} \epsilon_{\vec{k}}^\text{tb} \adjop{c}_{\vec{k}}\oper{c}_{\vec{k}} + \hbar \omega_0\sum_{\vec{q}}  \left( \adjop{b}_{\vec{q}} \oper{b}_{\vec{q}} \right)
\end{equation}
where we have ignored the ground-state phonon energy and where
\begin{equation}
    \epsilon_\vec{k}^\text{tb} = -t \sum_{    \delta} \cos(\vec{k}\cdot\vec{r}_\delta)
\end{equation}
where $\delta$ is an index of
If we separate the polaron ket in its electron and phonon part
$    \ket{\vec{k}, {n_\vec{q}}} = \ket{\vec{k}}\ket{n_1\dots n_\vec{q} \dots}$ we can find its unperturbed energy
\begin{equation}
    E_{\vec{k}, {n_\vec{q}}}^{(0)} = -t \sum_{    \delta} \cos(\vec{k}\cdot\vec{r}_\delta) + \hbar\omega_0 \sum_\vec{q} n_\vec{q}
\end{equation}
