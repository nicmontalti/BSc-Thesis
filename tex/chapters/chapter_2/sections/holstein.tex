\section{Holstein polarons}
Holstein proposed a different Hamiltonian to model polarons. The main difference is in how the lattice is treated. Instead of working with a continuum, polarizable medium like Fröhlich, Holstein took into account the discreteness of the lattice.

The model is based on the tight-binding model, which was briefly introduced in \cref{sec:tight-binding}. In the next sections, we will use it to derive Holstein Hamiltonian for a 1D system, and solve it in the weak coupling limit.

\subsection{Derivation of the Holstein Hamiltonian}
We now use the tight-binding model to sketch the derivation of the Holstein Hamiltonian. This Hamiltonian is used to model small polarons, since it treats the polarizable medium as a lattice.

To simplify the derivation, we consider a linear chain of $N$ atoms. At equilibrium, the atom $n$ is placed at the position $R_n = na$, where $a$ is the lattice constant. The atoms are allowed to move, and their interaction energy is given by the harmonic oscillator approximation. The Hamiltonian of the lattice is then
\begin{equation}
    \oper{H_{I-I}} = \sum_n \left( \frac{{P}_n^2}{2M} + \oh M\omega_0^2x_n^2 \right)
\end{equation}
where $P_n$ is the momentum of the $n^\text{th}$ atom, $x_n$ the separation between the atoms $n$ and $n+1$ and $M$ the mass.

The electrons are allowed to interact with the lattice through a potential
\begin{equation}
    U = \sum_n U(r-R_n, x_n)
\end{equation} The key feature of this model is that $U$ depends on the interatomic separation $x_n$. This results in the coupling of the electrons with the lattice vibrations, so with phonons. Adding the electron kinetic energy term, the total Hamiltonian is given by
\begin{equation}
    \oper{H} = \sum_n \left( -\frac{\hslash^2}{2M} \frac{\partial^2}{\partial x_n^2} + \oh M\omega_0^2x_n^2 \right)
    -\frac{\hslash^2}{2m} \frac{\partial^2}{\partial r^2}
    +     \sum_n U(r-R_n, x_n)
\end{equation}
Following the tight-binding approach described in the previous section, we express the electrons wavefunctions as a linear combination of the single atomic orbitals
\begin{equation}
    \psi(r) = \sum_{n} \alpha_{n}(x_1 \dots x_N) \phi_n(r-na, x_n)
\end{equation}
where $\alpha_n$ complex coefficients and $\phi_n$ are the solution of the corresponding \sche
\begin{equation}
    \left[ -\frac{\hslash^2}{2m} \frac{\partial^2}{\partial r^2} + U(r-na, x_n) \right] \phi_n = E_n(x_n) \phi_n
\end{equation}

This result in a differential equation for the coefficients $\alpha_n$, which is obtained through a standard projection procedure. The full calculation can be found in the Appendix of the original Holstein paper \cite{holstein1959}. We only report the result
\begin{multline} \label{eq:holstein_diff}
    \left[ i\hslash \pdv{}{t} - \sum_p \left( -\frac{\hslash^2}{2M} \pdv[2]{}{x_p} + \oh M \omega_0^2 x_p^2 \right) - E(x_n) - W_n(x_1\dots x_N) \right] \alpha_n(x_1\dots x_N)
    \\= \sum_{(\pm)} J(x_n, x_{n+1}) \alpha_{n\pm1}(x_1\dots x_N)
\end{multline}
where
\begin{align}
    W_n(x_1 \dots x_N) & = \int |\phi_n(r-na, x_n)|^2  \sum_{p \neq n} U(r-pa, x_p)  \differential r \\
    J(x_n, x_m)        & = \int \phi_n^*(r-na, x_n) U(r-na, x_n) \phi(r-ma, x_m) \differential r
\end{align}

We can simplify \cref{eq:holstein_diff} introducing three approximations:
\begin{enumerate}
    \item The neglect of the energies $ W_n(x_1 \dots x_N)$.
    \item The neglect of the $x$-dependance of $J$. The function $J(x_n, x_m)$ reduces to a constant $-J$, which can also be denominated $-t$. It corresponds to the parameter $t$ introduced in the tight-binding approximation in \cref{eq:tight_hamiltonian_position}.
    \item The $x$-dependance of the energy $E_n(x_n)$ is taken to be linear: $    E_n(x_n) = - A x_n$
\end{enumerate}
The physical meaning of the first approximation (1) is to neglect the perturbation of the wavefunction localized on the site $n$ caused by the interaction with other sites $p$. This assumption is reasonable in the tight-binding model that we are employing. The result is that the expectation value of the energy of an electron is only dependent on one coordinate $x_n$, the one of the site where the electron is localized.

With the previous simplifications, \cref{eq:holstein_diff} becomes
\begin{equation} \label{eq:holstein_diff_simp}
    i\hslash\pdv{}{t} \alpha_n = \sum_p\left( -\frac{\hslash^2}{2M} \pdv[2]{}{x_p} + \oh M \omega_0^2 x_p^2 \right) \alpha_n - t (\alpha_{n+1}+\alpha_{n-1})\alpha_n
    - A x_n \alpha_n
\end{equation}
A study of \cref{eq:holstein_diff_simp} in its current form is presented by Holstein in his paper \cite{holstein1959}. However, we will follow a different approach, expressing it in the second-quantization formalism. From now on, we also consider again a 3D system, generalizing the results that we obtain for the 1D one.

The first term of \cref{eq:holstein_diff_simp} is exactly the lattice interaction term we have already encountered in \cref{eq:lattice_hamiltonian}. In the momentum space, its second-quantized form is given by \cref{eq:phonon_hamiltonian}
\begin{equation}
    \oper{H}_\text{ph} = \hslash \omega_0\sum_{\vec{q}}  \left( \adjop{b}_{\vec{q}} \oper{b}_{\vec{q}} + \frac{1}{2} \right)
\end{equation}
where we have assumed the phonons to be dispersionless and optical, as we did for the Fröhlich Hamiltonian. The second term is the tight-binding energy encountered in \cref{sec:tight-binding} and its second-quantized form is give by \cref{eq:tight_hamiltonian_momentum}
\begin{equation}
    \oper{H}_\text{el} = \sum_{\vec{k}} \epsilon_{\vec{k}}^\text{tb} \adjop{c}_{\vec{k}}\oper{c}_{\vec{k}}
\end{equation}
Finally, the last term is responsible for the electron-phonon interaction energy. We have seen that, for a normal scattering process, the general form of the interaction matrix is given by \cref{eq:phonon_matrix_normal}. Since in this case the interaction amplitude is independent of the momentum, we can write the interaction matrix as constant $g$ in the position space and $g / \sqrt{N}$ in the momentum space. The interaction Hamiltonian is then given by \cref{eq:general_electron_phonon_hamiltonian}
\begin{equation}
    \oper{H}_\text{el-ph} = \frac{g}{\sqrt{N}}\sum_{\vec{k}, \vec{q}} \adjop{c}_{\vec{k}+\vec{q}} \oper{c}_\vec{k} (\adjop{b}_{-\vec{q}} + \oper{b}_\vec{q})
\end{equation}
The final Holstein Hamiltonian is
\begin{equation}
    \oper{H}_\text{Holstein} = \sum_{\vec{k}} \epsilon_{\vec{k}}^\text{tb} \adjop{c}_{\vec{k}}\oper{c}_{\vec{k}} + \hslash \omega_0\sum_{\vec{q}}  \left( \adjop{b}_{\vec{q}} \oper{b}_{\vec{q}} + \frac{1}{2} \right) + \frac{g}{\sqrt{N}}\sum_{\vec{k}, \vec{q}} \adjop{c}_{\vec{k}+\vec{q}} \oper{c}_\vec{k} (\adjop{b}_{-\vec{q}} + \oper{b}_\vec{q})
\end{equation}

\subsection{Weak coupling limit}
Like the Fröhlich Hamiltonian, the Holstein Hamiltonian is not exactly solvable in its general form. Although it is exactly solvable for a two state system \cite{tayebi2016}, we will use perturbation theory to find an approximated solution. We will follow the same procedure we adopted in \cref{sec:weak_frohlich}. This method is valid as long as the coupling constant
\begin{equation}
    \alpha = \frac{g^2}{z\hslash\omega_0t}
\end{equation}
where $z$ is the dimensionality of the problem, is small.

We start by splitting the Holstein Hamiltonian in a unperturbed term $\oper{H}^{(0)} = \oper{H}_\text{el} + \oper{H}_\text{ph}$ and the perturbation $\oper{H}_\text{el-ph}$.
The solution to the unperturbed problem is found solving the Hamiltonian
\begin{equation}
    \oper{H^{(0)}} = \sum_{\vec{k}} \epsilon_{\vec{k}}^\text{tb} \adjop{c}_{\vec{k}}\oper{c}_{\vec{k}} + \hslash \omega_0\sum_{\vec{q}}  \left( \adjop{b}_{\vec{q}} \oper{b}_{\vec{q}} \right)
\end{equation}
where we have ignored the ground-state phonon energy and where
\begin{equation} \label{eq:holstein_weak_free}
    \epsilon_\vec{k}^\text{tb} = -t \sum_{    \delta} \cos(\vec{k}\cdot\vec{r}_\delta)
\end{equation}
where $\delta$ is an index of the nearest neighbors. Working in one dimension, considering a linear chain of atoms separated by a unitary distance, \cref{eq:holstein_weak_free} becomes
\begin{equation} \label{eq:holstein_ground}
    \epsilon_k^\text{tb} = -2t \cos(k)
\end{equation}
where the factor two arises from the fact that there are two nearest neighbors for each ion.

If we separate the polaron ket in its electron and phonon part
$    \ket{k, n_q} = \ket{k}\ket{n_1\dots n_q \dots}$ we can find its unperturbed energy
\begin{equation}
    E_{k, n_q}^{(0)} = -t \sum_{    \delta} \cos(k) + \hslash\omega_0 \sum_q n_q
\end{equation}

We are now ready to add the perturbation
\begin{equation}
    \oper{H}_\text{el-ph} = \frac{g}{\sqrt{N}}\sum_{k, q} \adjop{c}_{k+q} \oper{c}_k (\adjop{b}_{-q} + \oper{b}_q)
\end{equation}
The form of this Hamiltonian is totally analogous to the Fröhlich Hamiltonian we have encountered in \cref{sec:weak_frohlich}, with $\frac{g}{\sqrt{N}}$ in place of $M_\vec{q}$. With analogous considerations, we can than conclude that the only states that give rise to non-null terms in the second-order energy correction
\begin{equation} \label{eq:holstein_weak}
    \Delta E_k^{(2)} = \sum_{a\neq{\{k;0\}}} \frac{\left|\bra{k;0}\oper{H}_\text{el-ph}\ket{a}\right|^2}{E_{\{k;0\}} - E_a}
    = \frac{g^2}{N} \sum_{a\neq{\{k;0\}}} \frac{\left|\bra{k;0} \adjop{c}_{k}\oper{c}_{k-q} (\oper{b}_{q} + \adjop{b}_{-q})\ket{a}\right|^2}{E_{\{k;0\}} - E_a}
\end{equation}
are states with an electron of momentum $k-q$ and a single phonon of momentum $q$. Using \cref{eq:holstein_ground} we can find the energy of the ground state
\begin{equation} \label{eq:energy_no_phonons}
    E_{k, 0}^{(0)} = -2t \cos(k)
\end{equation}
and of the excited state
\begin{equation} \label{eq:energy_one_phonon}
    E_a = E_{k, n_q=1}^{(0)} = -2t \cos(k-q) + \hslash\omega_0
\end{equation}
Plugging \cref{eq:energy_no_phonons,eq:energy_one_phonon} in \cref{eq:holstein_weak} and replacing the sum with an integral as we did in \cref{sec:weak_frohlich}, we find
\begin{multline}
    \Delta E_k = -  \frac{1}{2\pi}\int \differential q \frac{g^2}{2t\cos(k) +2t\cos(k-q) - \hslash \omega_0}
    \\ = -  \frac{1}{2\pi}\int \differential q \frac{\alpha \hslash \omega_0 t}{2y\cos(k) +2t\cos(k-q) - \hslash \omega_0 }
\end{multline}
The total energy is then
\begin{equation}
    E_k = -2t\cos(k)  - \frac{1}{2\pi}\int \differential q \frac{\alpha \hslash \omega_0 t}{2t\cos(k) +2t\cos(k-q) - \hslash \omega_0 }
\end{equation}