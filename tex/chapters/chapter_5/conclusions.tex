\chapter{Conclusions}
We started by introducing polarons at a general level, following the main historical steps that brought to the present understanding of these quasi-particles. Landau was the first to propose the concept of an electron localized in a self-generated potential well. In collaboration with Pekar, they proposed a first model to describe this new quasi-particle. Fröhlich and Holstein expanded their work proposing a description of polarons based on a quantum treatment of the lattice polarization. Fröhlich and Holstein Hamiltonians are currently used for the characterization of large and small polarons, respectively. These two types of polarons are characterized by the size of the polarization region, respectively larger and smaller than the unit cell. They also differ in their properties. In particular, we focused on the depth of their energy level and their mobility. The new energy level is formed $\simeq \SI{10}{meV}$ below the conduction band for large polarons and $\simeq \SI{1}{eV}$ below the conduction band for small polaron. The mobility is larger and decreasing with temperature for large polarons and smaller and increasing with temperature for small polarons.

In the second part of the thesis, we focused on numerical techniques useful for the simulation of small polarons. We presented the DFT formalism, describing its advantages and its flaws. A correction of it, DFT+U, was presented as well, together with its implementation in the Vienna Ab Initio Simulation Package. VASP was used for the  simulation of a small polaron trapped in rutile. The calculation was described in detail, focusing on the method used to localize the electron at the centre of the supercell. The results were then briefly discussed. A new energy level was found \SI{0.70}{eV} below the conduction band, and the charge of the extra electron was found to be localized on the atom at the centre of the supercell. This solution was compared with the solution found for an extra electron delocalized in the cell. The delocalized electron enters the conduction band and its charge is delocalized on the whole cell.

The results were compatible with the expectations, at least at a qualitative level. The simulation could be improved proposing a quantitative comparison with the literature. The band structure of the supercell could be reduced to the unit cell one applying band-unfolding techniques. The effective mass and the formation energy of the polaron could be computed as well. With compatible results, the work could be extended to the simulation of a small polaron in a new material, trying to determine if it is prone to the formation of polarons or not.