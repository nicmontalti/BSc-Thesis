\selectlanguage{italian}
\begin{abstract}
    Lo scopo di questa tesi è di introdurre il concetto di polarone e di presentare un calcolo DFT di uno small polaron nel rutile. Nei primi capitoli si analizzano gli small e large polarons con un approccio analitico, basato sulle Hamiltoniane di Holstein e di Fröhlich. Il formalismo matematico e le basi fisiche necessarie vengono introdotte nel primo capitolo.

    Nella seconda parte della tesi, si introduce la Density Functional Theory, una sua correzione (DFT+U) e la sua implementazione in VASP. Il calcolo numerico è poi descritto e discusso a un livello qualitativo. La soluzione polaronica è confrontata con un elettrone delocalizzato nel materiale.

    Il calcolo ha evidenziato come il polarone crea un nuovo stato energetico \SI{0.70}{eV} sotto la banda di conduzione. Il nuovo livello energetico era visibile sia nella struttura a bande, sia nel grafico della densità degli stati. L'elettrone si localizza su un atomo di titanio, distorcendo il reticolo circostante. In particolare, i quattro atomi più vicini al titanio si allontano di \SI{0.085}{\angstrom}, mentre i due atomi di ossigeno più distanti di \SI{0.023}{\angstrom}.

    I risultati sono qualitativamente compatibili con la letteratura. Futuri sviluppi del lavoro possono cercare di migliorare la precisione del calcolo, proponendo un confronto quantitativo.
\end{abstract}

\selectlanguage{english}
\begin{abstract}
    The aim of this thesis is to introduce the polaron concept and to perform a DFT numerical calculation of a small polaron in rutile $\ce{TiO_2}$. In the first chapters we present an analytical study of small and large polarons, based on the Holstein and Fröhlich Hamiltonians. The necessary mathematical formalism and physics fundamentals are briefly reviewed in the first chapter.

    In the second part of the thesis, Density Functional Theory is introduced together with the DFT+U correction and its implementation in VASP. The actual calculation is then described and discussed at qualitative level. The polaronic solution is compared in detail with a delocalized electron.

    The calculation showed how the polaron creates a new energy level \SI{0.70}{eV} under the conduction band. The energy level was visible both in the band structure diagram and in the density of states diagram. The electron localizes on a Ti atom, distorting the surrounding lattice. In particular, the four oxygen atoms closer to the titanium atom are displaced by \SI{0.085}{\angstrom} outwards, whereas the two further oxygen atoms by \SI{0.023}{\angstrom}.

    The results are compatible, at a qualitative level, with the literature. Further developments of this work may try to improve the precision of the results and compare them with the literature.

\end{abstract}

\chapter*{Acknowledgements}
This work would have not been possible without the help and the support of many people, more than this page could contain. Not only the ones that have contributed directly or indirectly to it in the past months, but everyone I met along the path of taking this degree.

First of all, I want to thank my supervisor Prof. Cesare Franchini for giving me the opportunity of working on such an interesting topic. I am grateful for the time he dedicated to me and to this work. I also want to thank Lorenzo Varrassi for the help he gave me understanding DFT and VASP. I would not have been able to conclude this work without his advices. Finally, I am grateful to the Department of Physics of the University of Bologna for letting me use their HPC Cluster for running my calculations.

I want to thank my family, especially my parents and grandparents, for believing in me and encouraging my passion in Physics. This work is dedicated to my mother, who could not fulfill the dream of studying this subject like me. I thank my father for his patience, his support and for feeding my curiosity since I was child. I am grateful to my sisters, for bearing me talking about my work, and to my grandparents, for their support and interest in my studies.

Between my friends, a special thanks goes to Mattia, Adriano, Alessandro and Pietro, for making me feel home in a city it has never been really mine. I will never be able to repay them for their heartwarming hospitality. I have to thank Alessandro in particular for all the time he dedicated to listening to me, and Claudia and Lorenzo for their friendship and the time we spent together.

I also want to thank Lorenzo and Alice, who have been two invaluable peers in the path to obtaining this degree. They have had a major role in keeping my curiosity and my enthusiasm always high in the past years. They have been among the few people I could actually discuss my work with, and help me when I needed it.

A final, heartfelt thanks goes to Edoardo, Thomas, Jakob F., Jakob S., Samira, Riley, and all the other wonderful people I met in Groningen. With them, I spent the best months of my studies. Each of them changed me in their own way, making me discover a whole new me I could have not found otherwise. I am sure we will meet again many times all over Europe.

I hope that all these people will be as proud of this work as I am, even the ones who might not understand it.

\clearpage
\begingroup
\hypersetup{linkcolor=black}
\pdfbookmark[chapter]{\contentsname}{toc}
\tableofcontents
\listoffigures
\endgroup

\chapter{Introduction}
%\addcontentsline{toc}{chapter}{Introduction}
%\chaptermark{Introduction}
%\section*{Polarons: Historical overview}
The term polaron was first used by Solomon Pekar in 1946 to define an electron that localizes itself in a potential well, self-generated by the polarization of the material \cite{pekar1946}. The result can be described as an electron surrounded by a cloud of phonons. The electron moves in the crystal and the polarization follows it as it moves.

Lev Landau was the first to propose the concept of an auto-localized electron in a crystal in a 1933 paper \cite{terhaar1965}. The idea was then developed by Pekar, who considered a single electron interacting with a dielectric continuum medium \cite{pekar1946, pekar1947}. This interaction was shown to cause an enhancement of the effective mass and a localization of the wavefunction \cite{landau1948}. In their work, Landau and Pekar used a quantum mechanical description of the electron and a classical description of the medium. A full quantum mechanical description was then developed by Fr\"ohlich \cite{frohlich1950} and Holstein \cite{holstein1959}, who formalized the distinction between large and small polarons.

Fr\"ohlich considered an electron in a continuum, polarizable medium. The electron is assumed to interact only with longitudinal optical phonons. The interaction gives rise to the polarization of the material, which generates a potential well in which the electron localizes. Since the medium is treated as a continuum, the result is valid only for large polarons, that are polarons with an effective radius larger than the lattice constant. On the other hand, Holstein considered short-range electron-phonon interactions, resulting from the coupling between a carrier and the strain where it resides. Holstein theory takes into account the discreteness of the lattice, and it is used to describe small polarons, that are polarons with an effective radius smaller than the lattice constant.

All the attempts find analytical solutions to Fr\"ohlich Hamiltonian have been fruitless, and Holstein Hamiltonian is exactly solvable only in the two-site case \cite{rongsheng2002}. Approximation techniques and numerical simulations are unavoidable. Good results have been achieved with the diagrammatic quantum Monte Carlo method to solve both the Fr\"ohlich and Holstein Hamiltonians \cite{prokofev1998,mishchenko2000}. For small polarons, DFT+U methods have also proven to be applicable \cite{kokott2018}. A rigo­rous, ab initio computational theory of polarons was recently developed by Feliciano Giustino and colleagues combining the Landau–Pekar model with DFT \cite{sio2019}.

\section*{Outline}
The aim of this thesis is to give an introductory analytical description of polarons and to perform a numerical simulation of a small polaron in rutile. The focus of the analytical discussion is to derive Holstein and Fröhlich Hamiltonians. The results of the numerical simulation are discussed at a qualitative level.

Polarons originate from the coupling of electrons with phonons in crystals. To give a complete description of these quasi-particles is necessary to introduce a mathematical formalism that allows us to describe such a big number of particles and their interactions. Second quantization proved to be a good choice for various many-body problems in solid state physics, polarons included.

In \cref{ch:review}, we will start by reviewing the second quantization formalism, and we will use it to describe electrons and phonons in crystals. Electrons and phonons are firstly studied as non-interacting systems. Then, their interaction, essential to describe polarons, is investigated as well.

In \cref{ch:polarons} the theory of polarons is explained starting from the original Landau-Pekar model. Then, Fr\"ohlich and Holstein Hamiltonians are derived and solved in the small and large coupling regimes using perturbation theory and variational principles. Eventually, small and large polarons are compared.

In \cref{ch:dft} Density Functional Theory (DFT) is presented. The theory behind it is explained together with some of its problems. An extension of it, DFT+U, is also introduced to solve some inaccuracies. Lastly, its implementation is discussed, focusing on VASP, the Vienna Ab-initio Simulation Package.

In \cref{ch:simulation} the simulation of a small polaron in $\ce{TiO_2}$ is presented. The whole simulation process is described, and the results are discussed. The polaronic solution is compared to a delocalized one, with particular emphasis one the density of states, band structure and charge isosurfaces.