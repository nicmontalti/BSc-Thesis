\newgeometry{textwidth=450pt}
\begin{titlepage}
    %
    %
    % UNA VOLTA FATTE LE DOVUTE MODIFICHE SOSTITUIRE "RED" CON "BLACK" NEI COMANDI \textcolor
    %
    %


    \begin{center}
        {{\Large{\textsc{Alma Mater Studiorum $\cdot$ Universit\`a di Bologna}}}}
        \rule[0.1cm]{15.8cm}{0.1mm}
        \rule[0.5cm]{15.8cm}{0.6mm}
        \\\vspace{3mm}

        {\small{\bf Scuola di Scienze \\
                Dipartimento di Fisica e Astronomia\\
                Corso di Laurea in Fisica}}

    \end{center}

    \vspace{23mm}
    \centering
    \begin{center}%\textcolor{red}{
        %
        % INSERIRE IL TITOLO DELLA TESI
        %
        {\LARGE{\bf Analytical and numerical study of polarons using Density Functional Theory}}\\
        %}
    \end{center}

    \vspace{50mm} \par \noindent

    \begin{minipage}[t]{0.47\textwidth}
        {\large{\bf Relatore:
                \vspace{2mm}\\
                Prof. Cesare Franchini}} \\\\
        %
        % INSERIRE IL NOME DEL CORRELATORE CON IL RELATIVO TITOLO DI DOTTORE O PROFESSORE
        %
        % SE NON AVETE UN CORRELATORE CANCELLATE LE PROSSIME 3 RIGHE
        %
        %\textcolor{red}{
        \bf Correlatore:
        \vspace{2mm}\\
        Dott. Lorenzo Varrassi\\\\
    \end{minipage}
    %
    \hfill
    %
    \begin{minipage}[t]{0.47\textwidth}\raggedleft
        {\large{\bf Presentata da:
                \\
                \vspace{2mm}
                %
                % INSERIRE IL NOME DEL CANDIDATO
                %
                Nicolò Montalti}}
    \end{minipage}

    \vspace{40mm}

    \begin{center}
        %
        % INSERIRE L'ANNO ACCADEMICO
        %
        Anno Accademico \textcolor{black}{2021/2022}
    \end{center}

\end{titlepage}
\restoregeometry